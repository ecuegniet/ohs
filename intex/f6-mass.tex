%%%%%%%%%%%%%%%%%%%%%%%%%%%%%%%%%%%%%%%%%%%%%%%%%%%%%%%%%%%%%%%%%%%%%%%%%%%%%%%%%%%%%%%%%%%%%%%%
%%%%%%%%%%%%%%%%%%%%%%%%%%%%%%%%%%%%%%%%%%%%%%%%%%%%%%%%%%%%%%%%%%%%%%%%%%%%%%%%%%%%%%%%%%%%%%%%
\sloppy
\grechangecount{endofwordpenalty}{-2000}

\ohschapter{Ad Liturgiam Verbi}

%%%%%%%%%%%%%%%%%%%%%%%%%%%%%%%%%%%%%%
\ohslectio{le-f6-mass1}
\separator{3}

\ohsproprium{Tractus}{II}{tr-domine_exaudi-b}
\separator{3}

\ohslectio{le-f6-mass2}
\separator{3}

\ohsproprium{Graduale}{V}{gr-christus_factus}

\separator{3}

\ohslectio{le-f6-mass3}
\separator{4}

%%%%%%%%%%%%%%%%%%%%%%%%%%%%%%%%%%%%%%
\nadpisb{Oratio universalis}
\begin{footnotesize}
\nadpisf{I. ZA SVÄTÚ CIRKEV}

Modlime sa, milovaní bratia a sestry, 
za svätú Božiu Cirkev: nech jej náš Boh a Pán 
láskavo udelí pokoj, jednotu 
a ochranu na celom svete, 
aby sme mohli pokojne a nerušene oslavovať
Boha Otca všemohúceho. 

Všemohúci a večný Bože, 
ty si v Kristovi zjavil svoju slávu všetkým národom; 
ochraňuj dielo svojej lásky, 
aby Cirkev, rozšírená po celom svete, 
vytrvala v pevnej viere a neohrozene ťa vyznávala. 
Skrze Krista, nášho Pána. Amen.

\nadpisf{II. ZA PÁPEŽA}

Modlime sa za nášho Svätého Otca,
ktorého si náš Boh a Pán vyvolil spomedzi biskupov:  
nech ho chráni a zachová pre svoju Cirkev,  
aby mohol spravovať svätý ľud Boží. 

Všemohúci a večný Bože, 
od tvojej vôle závisí všetko; 
láskavo vypočuj naše prosby 
a ochraňuj nášho najvyššieho pastiera, 
aby kresťanský ľud, ktorý si mu zveril, 
pod jeho vedením rástol vo viere a láske. 
Skrze Krista, nášho Pána. Amen.

\nadpisf{III. ZA BISKUPOV, KŇAZOV, DIAKONOV A VŠETOK VERIACI ĽUD}

Modlime sa za nášho biskupa, 
za všetkých biskupov, kňazov a diakonov svätej Cirkvi 
za všetok veriaci ľud. 

Všemohúci a večný Bože,  
tvoj Duch posväcuje a spravuje celú Cirkev;  
vypočuj naše prosby za tvojich služobníkov  
a udeľ im milosť, aby ti všetci verne slúžili.  
Skrze Krista, nášho Pána.    
Amen.

\nadpisf{IV. ZA KATECHUMENOV}

Modlime sa za tých, čo sa pripravujú na krst: nech  
náš Boh a Pán pripraví ich srdcia a otvorí im náruč
svojho milosrdenstva, aby v prameni znovuzrodenia  
dosiahli odpustenie hriechov a nový život v Ježišovi  
Kristovi, našom Pánovi. 

Všemohúci a večný Bože, 
ty stále požehnávaš svoju Cirkev 
novým potomstvom; 
zveľaďuj dar viery a poznania v tých, 
čo sa pripravujú na krst, 
aby sa znovuzrodili v krstnom prameni 
a stali sa tvojimi deťmi. 
Skrze Krista, nášho Pána.
Amen.

\nadpisf{V. ZA JEDNOTU KRESŤANOV}

Modlime sa za všetkých bratov, 
čo veria v Krista 
a úprimne žijú podľa pravdy: 
nech ich náš Boh a Pán zjednotí 
a zachová vo svojej jedinej Cirkvi. 

Všemohúci a večný Bože, 
ty zjednocuješ rozdelených 
a zjednotených zachovávaš vo svornosti 
láskavo zhliadni na všetkých, 
ktorí veria v tvojho Syna, 
a keďže nás posvätil ten istý krst, 
nech nás spája aj pravá viera a vzájomná láska. 
Skrze Krista, nášho Pána.
Amen.

\nadpisf{VI. ZA ŽIDOV}

Modlime sa aj za synov židovského národa,  
lebo oni boli prví, ku ktorým Pán prehovoril:  
nech vzrastajú v láske k Bohu  
a vo vernosti k jeho zákonu. 

Všemohúci a večný Bože, 
ty si dal svoje prisľúbenia 
Abrahámovi a jeho potomkom; 
milostivo vypočuj prosby svojej Cirkvi za národ, 
ktorý bol tvojím vyvoleným ľudom, 
aby dosiahol plné vykúpenie. 
Skrze Krista, nášho Pána.
Amen.

\nadpisf{VII. ZA TÝCH, ČO NEVERIA V KRISTA}

Modlime sa za tých, 
čo neveria v Krista, 
aby ich Duch Svätý osvietil  
a priviedol na cestu spásy. 

Všemohúci a večný Bože, 
daj, aby tí, čo nevyznávajú Krista, 
žili pred tvojou tvárou statočným životom, 
a tak došli k pravde; 
nám však pomôž 
vždy hlbšie vnikať do tajomstiev tvojho života  
a rásť vo vzájomnej láske, 
aby sme boli vo svete vždy 
dokonalejšími svedkami 
tvojej dobroty. 
Skrze Krista, nášho Pána.
Amen.

\nadpisf{VIII. ZA TÝCH, ČO NEVERIA V BOHA}

Modlime sa za tých, čo neveria v Boha, 
aby úprimne žili podľa svedomia, 
a tak došli k poznaniu pravého Boha.

Všemohúci a večný Bože, 
ty si vložil do srdca človeka 
takú silnú túžbu po tebe, 
že sa uspokojí len vtedy, keď ťa nájde; 
prosíme ťa, daj, nech všetci pocítia prejavy tvojej lásky  
a povzbudia sa na príkladnom živote tvojich veriacich,  
aby napriek všetkým prekážkam a ťažkostiam uznali, 
že ty jediný si pravý Boh a Otec všetkých ľudí. 
Skrze Krista, nášho Pána.
Amen.

\nadpisf{IX. ZA TÝCH, ČO SPRAVUJÚ ŠTÁT}

Modlime sa za tých, čo spravujú štát:  
nech náš Boh a Pán vedie ich mysle a srdcia,  
aby podľa jeho vôle pracovali za pravý pokoj  
a slobodu pre všetkých. 

Všemohúci a večný Bože, 
v tvojich rukách sú ľudské srdcia i práva národov; 
dobrotivo pomáhaj tým, 
čo nás zákonite spravujú, 
aby všade presadzovali 
pravý pokoj, blahobyt 
ľudu a náboženskú slobodu. 
Skrze Krista, nášho Pána.
Amen.

\nadpisf{X. ZA TÝCH, ČO ZNÁŠAJÚ ÚTRAPY}

Modlime sa, milovaní bratia a sestry, 
k Bohu Otcu všemohúcemu, 
aby oslobodil svet od všetkých neporiadkov, 
odvrátil choroby, zahnal hlad, 
oslobodil nevinne väznených, 
ujal sa utláčaných, cestujúcim doprial bezpečnosť, 
vzdialeným z domova šťastný návrat, chorým zdravie  
a umierajúcim večnú spásu.

Všemohúci a večný Bože, 
útecha zarmútených a posila trpiacich, 
vypočuj prosby tých, čo ťa vzývajú vo svojich súženiach, 
a poteš ich v každej núdzi svojou láskavou pomocou. 
Skrze Krista, nášho Pána.   
Amen.

\end{footnotesize}
%%%%%%%%%%%%%%%%%%%%%%%%%%%%%%%%%%%%%%
\nadpisb{Adoratio Sanctæ Crucis}
\nadpisb{Invitatio in ostendenda sancta cruce}

\ohsannotation{}{VI}
\gregorioscore{./ant/ant-ecce_lignum_crucis}

\begin{minipage}[t]{0.15\textwidth}%
%\vspace*{\fill}
\textit{Omnes:}
%\vspace*{\fill}
\end{minipage}%
%\begin{array}{l}
\gresetinitiallines{0}
\begin{minipage}[t]{0.85\textwidth}%
	\gregorioscore{./ant/ant-ecce_lignum_crucis--omnes}
\end{minipage}%
%\end{array}
\vspace{1em}
\ohstranslate{ant-ecce_lignum_crucis}

%%%%%%%%%%%%%%%%%%%%%%%%%%%%%%%%%%%%%%
\nadpisb{Cantus in adoratione s. crucis peragendi}
\gresetinitiallines{1}
\ohsannotation{Ant.}{IV}
\gregorioscore{./mass/ant-crucem_tuam}
\textit{Et repetitur antiphona} Crucem tuam.

\ohstranslate{ant-crucem_tuam}

%%%%%%%%%%%%%%%%%%%%%%%%%%%%%%%%%%%%%%
\nadpisb{Improperia}
\nadpisb{I}
\gresetinitiallines{1}
\ohsannotation{}{\Vbar}
\gregorioscore{./mass/im-popule_meus}
\ohstranslate{im-popule_meus}

%%%%%%%%%%%%%%%%%%%%%%%%%%%%%%%%%%%%%%
\begin{minipage}[t]{0.55\textwidth}%
	\gregorioscore{./mass/im-hagios_o_theos}
\end{minipage}%
\begin{minipage}[t]{0.45\textwidth}%
	\gregorioscore{./mass/im-sanctus_deus}
\end{minipage}%

\vspace{1em}
\begin{minipage}[t]{0.55\textwidth}%
	\gregorioscore{./mass/im-hagios_ischyros}
\end{minipage}%
\begin{minipage}[t]{0.45\textwidth}%
	\gregorioscore{./mass/im-sanctus_fortis}
\end{minipage}%

\vspace{1em}
\gregorioscore{./mass/im-hagios_athanatos}

\vspace{1em}
\gregorioscore{./mass/im-sanctus_immortalis}

\ohstranslate{trishagion}
%%%%%%%%%%%%%%%%%%%%%%%%%%%%%%%%%%%%%%
\vspace{1em}
\gregorioscore{./mass/im-quia_eduxi_te}
\ohstranslate{im-quia_eduxi_te}
%%%%%%%%%%%%%%%%%%%%%%%%%%%%%%%%%%%%%%
\vspace{1em}
\gregorioscore{./mass/im-quid_ultra_debui}
\ohstranslate{im-quid_ultra_debui}
%%%%%%%%%%%%%%%%%%%%%%%%%%%%%%%%%%%%%%
\pagebreak[4]
\nadpisb{II}
\ohsformatverses
\grechangedim{spacelinestext}{5.5 mm}{scalable}
\gresetinitiallines{1}
\ohsannotation{}{\Vbar}
\gregorioscore{./mass/im-ego_propter_te}
\ohstranslate{im-ego_propter_te}
\begin{small}\textit{Chorus repetit:}\end{small}
\ohsformatresponsorium
\gregorioscore{./mass/im-popule_meus2}
\ohstranslate{im-popule_meus2}

\vspace{1em}
\ohsformatverses
\gresetinitiallines{0}
\gregorioscore{./mass/im-ego_te_eduxi}
\ohstranslate{im-ego_te_eduxi}

\vspace{1em}
\gregorioscore{./mass/im-ego_ante_te}
\ohstranslate{im-ego_ante_te}

\vspace{1em}
\gregorioscore{./mass/im-ego_ante_te_praeivi}
\ohstranslate{im-ego_ante_te_praeivi}

\vspace{1em}
\gregorioscore{./mass/im-ego_te_pavi}
\ohstranslate{im-ego_te_pavi}

\vspace{1em}
\gregorioscore{./mass/im-ego_te_potavi}
\ohstranslate{im-ego_te_potavi}

\vspace{1em}
\gregorioscore{./mass/im-ego_propter_te_chananaeorum}
\ohstranslate{im-ego_propter_te_chananaeorum}

\vspace{1em}
\gregorioscore{./mass/im-ego_dedi}
\ohstranslate{im-ego_dedi}

\vspace{1em}
\gregorioscore{./mass/im-ego_te_exaltavi}
\ohstranslate{im-ego_te_exaltavi}

\separator{4}

%%%%%%%%%%%%%%%%%%%%%%%%%%%%%%%%%%%%%%
\ohsformatresponsorium
\gresetinitiallines{1}
\ohsannotation{Hymn.}{I}
\gregorioscore{./mass/hy-crux_fidelis1}

\ohsannotation{}{I}
\gregorioscore{./mass/hy-crux_fidelis2}

\begin{small}\textit{Repetitur} Crux fidelis \textit{usque ad} * Dulce lignum.\end{small}
\gresetinitiallines{0}
\gregorioscore{./mass/hy-crux_fidelis3}
%%%%%%%%%%%%%%%%%%%%%%%%%%%%%%%%%%%%%%

%\vspace{1em}
\begin{center}\greseparator{2}{20}\end{center}
%%%%%%%%%%%%%%%%%%%%%%%%%%%%%%%%%%%%%%%%%%%%%%%%%%%%%%%%%%%%%%%%%%%%%%%%%%%%%%%%%%%%%%%%%%%%%%%%
%%%%%%%%%%%%%%%%%%%%%%%%%%%%%%%%%%%%%%%%%%%%%%%%%%%%%%%%%%%%%%%%%%%%%%%%%%%%%%%%%%%%%%%%%%%%%%%%

