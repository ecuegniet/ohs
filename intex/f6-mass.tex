%%%%%%%%%%%%%%%%%%%%%%%%%%%%%%%%%%%%%%%%%%%%%%%%%%%%%%%%%%%%%%%%%%%%%%%%%%%%%%%%%%%%%%%%%%%%%%%%
%%%%%%%%%%%%%%%%%%%%%%%%%%%%%%%%%%%%%%%%%%%%%%%%%%%%%%%%%%%%%%%%%%%%%%%%%%%%%%%%%%%%%%%%%%%%%%%%
\sloppy
\grechangecount{endofwordpenalty}{-2000}

\ohschapter{Celebratio passionis Domini}
\nadpisb{Ad Liturgiam Verbi}

%%%%%%%%%%%%%%%%%%%%%%%%%%%%%%%%%%%%%%
\ohslectio{le-f6-mass1}
\separator{3}

\ohsproprium{Tractus}{II}{tr-domine_exaudi-b}
\separator{3}

\ohslectio{le-f6-mass2}
\separator{3}

\ohsproprium{Graduale}{V}{gr-christus_factus}

\separator{3}

\ohslectio{le-f6-mass3}
\separator{4}

%%%%%%%%%%%%%%%%%%%%%%%%%%%%%%%%%%%%%%
\nadpisb{Oratio universalis}
\begin{footnotesize}
\nadpisf{I. POUR LA SAINTE ÉGLISE}

Prions, mes très chers frères, pour la sainte Église de Dieu et supplions le Seigneur de lui donner la paix et l’union et de la garder par toute la terre, en lui assujettissant les principautés et les puissances ; qu’il nous accorde une vie calme et tranquille, afin que nous glorifiions à jamais Dieu le Père tout-puissant.

Dieu tout-puissant et éternel, qui par le Christ avez révélé votre gloire à toutes les nations, conservez l’œuvre de votre miséricorde, afin que votre Église, répandue par toute la terre, persévère avec une ferme foi dans la confession de votre Nom. Par le même.

\nadpisf{II. POUR LE PAPE}

Prions aussi pour notre Saint-Père le Pape N…, afin que le Seigneur notre Dieu, qui l’a élu dans l’ordre de l’épiscopat, lui conserve la santé pour le bien de sa sainte Église et pour la conduite du saint peuple de Dieu.

Dieu tout-puissant et éternel, qui faites subsister toutes choses par votre sagesse, recevez favorablement nos prières et conservez, par votre bonté, le Pontife que vous nous avez choisi ; que le peuple chrétien qu’il gouverne par votre autorité, croisse de plus en plus dans les mérites de la foi, sous la conduite d’un si grand Pontife. Par Notre-Seigneur.

\nadpisf{III. POUR LE CLERGÉ ET LES FIDÈLES}

Prions aussi pour tous les évêques, prêtres, diacres, sous-diacres, acolytes, exorcistes, lecteurs, portiers, confesseurs, vierges, veuves et pour tout le saint peuple de Dieu.

Dieu tout-puissant et éternel, dont l’Esprit sanctifie et gouverne tout le corps de l’Église, exaucez nos très humbles prières pour tous les ordres qu’elle renferme, afin que, par le secours de votre grâce, ces divers degrés soient fidèles dans votre service. Par Notre-Seigneur.

\nadpisf{IV. POUR LES CHEFS D'ÉTAT}

Prions aussi pour tous les chefs d’État et pour tous ceux qui partagent leur pouvoir et leurs responsabilités, afin que le Seigneur notre Dieu dirige leur esprit et leur cœur selon sa volonté, en vue de nous maintenir dans la paix.

Dieu tout-puissant et éternel, tous les pouvoirs et le gouvernement de tous les peuples sont entre vos mains : regardez avec bienveillance ceux qui exercent sur nous l’autorité, afin que sous la protetion de votre main, la religion chrétienne garde partout sa pureté, et que la patrie connaisse toujours la sécurité. Par Notre-Seigneur.

\nadpisf{V. ZPOUR LES CATÉCHUMÈNES}

Prions encore pour nos catéchumènes, afin que le Seigneur notre Dieu ouvre les oreilles de leur cœur et la porte de sa miséricorde ; et que, ayant reçu la rémission de tous leurs péchés dans le bain de la régénération, ils soient incorporés avec nous en Jésus-Christ Notre Seigneur.

Dieu tout-puissant et éternel, qui donnez sans cesse de nouveaux enfants à votre Église, accroissez la foi et l’intelligence de nos catéchumènes, afin que, étant régénérés dans la fontaine du baptême, ils soient admis au nombre de vos enfants adoptifs. Par Notre-Seigneur.

\nadpisf{VI. POUR LES BESOIN DES FIDÈLES}

Prions le Dieu le Père tout-puissant, mes très chers frères, qu’il veuille bien purger le monde de toute erreur, dissiper les maladies, chasser la famine, ouvrir les prisons, rompre les liens des captifs, accorder aux voyageurs un heureux retour, rendre la santé aux malades et accorder aux navigateurs un port salutaire.

Dieu tout-puissant et éternel, consolation des affligés et force de ceux qui sont dans la peine, laissez monter jusqu’à vous les cris et les prières de ceux qui vous invoquent dans leurs afflictions, afin qu’ils ressentent tous avec joie, dans leurs besoins, le secours de votre miséricorde. Par Notre-Seigneur.

\nadpisf{VII. POUR L'UNITÉ DE L'ÉGLISE}

Prions également pour les hérétiques et les schismatiques, afin que le Seigneur notre Dieu dissipe toutes leurs erreurs et daigne les ramener à notre Mère la sainte Église catholique et apostolique.

Dieu tout-puissant éternel, qui sauvez tous les hommes et qui ne voulez pas qu’aucun périsse, jetez les yeux sur les âmes qui ont été séduites par les artifices du démon, afin que, déposant la perversité hérétique, leurs cœurs égarés viennent à résipiscence et retournent à l’unité de votre vérité. Par Notre-Seigneur.

\nadpisf{VIII. POUR LA CONVERSION DES JUIFS}

Prions aussi pour les Juifs, afin que notre Dieu et Seigneur éclaire leurs cœurs, afin qu'ils reconnaissent Jésus-Christ, le Sauveur de tous les hommes.

Dieu tout-puissant et éternel, qui voulez que tous les hommes soient sauvés et parviennent à la connaissance de la vérité, accordez-nous avec bonté que, la plénitude des païens étant entrée dans votre Église, tout Israël soit sauvé.

\nadpisf{IX. POUR LA CONVERSION DES INFIDÈLES}

Prions enfin pour les païens, afin que le Dieu tout-puissant ôte l’iniquité de leurs cœurs et que, abandonnant leurs idoles, ils se convertissent au Dieu vivant et véritable et à son Fils unique, Jésus-Christ notre Dieu et notre Seigneur.

Dieu tout-puissant et éternel, qui ne voulez pas la mort, mais la vie des pécheurs, exaucez la prière que nous vous faisons en faveur des idolâtres ; délivrez-les du culte des idoles et donnez-leur place dans votre sainte Église pour l’honneur et la gloire de votre Nom.

\end{footnotesize}
%%%%%%%%%%%%%%%%%%%%%%%%%%%%%%%%%%%%%%
\nadpisb{Adoratio Sanctæ Crucis}
\nadpisb{Invitatio in ostendenda sancta cruce}

\ohsannotation{}{VI}
\gregorioscore{./ant/ant-ecce_lignum_crucis}

\begin{minipage}[t]{0.15\textwidth}%
%\vspace*{\fill}
\textit{Omnes:}
%\vspace*{\fill}
\end{minipage}%
%\begin{array}{l}
\gresetinitiallines{0}
\begin{minipage}[t]{0.85\textwidth}%
	\gregorioscore{./ant/ant-ecce_lignum_crucis--omnes}
\end{minipage}%
%\end{array}
\vspace{1em}
\ohstranslate{ant-ecce_lignum_crucis}

%%%%%%%%%%%%%%%%%%%%%%%%%%%%%%%%%%%%%%
\pagebreak
\nadpisb{Cantus in adoratione s. crucis peragendi}
\gresetinitiallines{1}
\ohsannotation{Ant.}{IV}
\gregorioscore{./mass/ant-crucem_tuam}
\textit{Et repetitur antiphona} Crucem tuam.

\ohstranslate{ant-crucem_tuam}

%%%%%%%%%%%%%%%%%%%%%%%%%%%%%%%%%%%%%%
\pagebreak
\nadpisb{Improperia}
\nadpisb{I}
\gresetinitiallines{1}
\ohsannotation{}{\Vbar}
\gregorioscore{./mass/im-popule_meus}
\ohstranslate{im-popule_meus}

%%%%%%%%%%%%%%%%%%%%%%%%%%%%%%%%%%%%%%
\begin{minipage}[t]{0.55\textwidth}%
	\gregorioscore{./mass/im-hagios_o_theos}
\end{minipage}%
\begin{minipage}[t]{0.45\textwidth}%
	\gregorioscore{./mass/im-sanctus_deus}
\end{minipage}%

\vspace{1em}
\begin{minipage}[t]{0.55\textwidth}%
	\gregorioscore{./mass/im-hagios_ischyros}
\end{minipage}%
\begin{minipage}[t]{0.45\textwidth}%
	\gregorioscore{./mass/im-sanctus_fortis}
\end{minipage}%

\vspace{1em}
\gregorioscore{./mass/im-hagios_athanatos}

\vspace{1em}
\gregorioscore{./mass/im-sanctus_immortalis}

\ohstranslate{trishagion}
%%%%%%%%%%%%%%%%%%%%%%%%%%%%%%%%%%%%%%
\vspace{1em}
\gregorioscore{./mass/im-quia_eduxi_te}
\ohstranslate{im-quia_eduxi_te}
%%%%%%%%%%%%%%%%%%%%%%%%%%%%%%%%%%%%%%
\vspace{1em}
\gregorioscore{./mass/im-quid_ultra_debui}
\ohstranslate{im-quid_ultra_debui}
%%%%%%%%%%%%%%%%%%%%%%%%%%%%%%%%%%%%%%
%\pagebreak[4]
\nadpisb{II}
\ohsformatverses
\grechangedim{spacelinestext}{5.5 mm}{scalable}
\gresetinitiallines{1}
\ohsannotation{}{\Vbar}
\gregorioscore{./mass/im-ego_propter_te}
\ohstranslate{im-ego_propter_te}
\begin{small}\textit{Chorus repetit:}\end{small}
\ohsformatresponsorium
\gregorioscore{./mass/im-popule_meus2}
\ohstranslate{im-popule_meus2}

\vspace{1em}
\ohsformatverses
\gresetinitiallines{0}
\gregorioscore{./mass/im-ego_te_eduxi}
\ohstranslate{im-ego_te_eduxi}

\vspace{1em}
\gregorioscore{./mass/im-ego_ante_te}
\ohstranslate{im-ego_ante_te}

\vspace{1em}
\gregorioscore{./mass/im-ego_ante_te_praeivi}
\ohstranslate{im-ego_ante_te_praeivi}

\vspace{1em}
\gregorioscore{./mass/im-ego_te_pavi}
\ohstranslate{im-ego_te_pavi}

\vspace{1em}
\gregorioscore{./mass/im-ego_te_potavi}
\ohstranslate{im-ego_te_potavi}

\vspace{1em}
\gregorioscore{./mass/im-ego_propter_te_chananaeorum}
\ohstranslate{im-ego_propter_te_chananaeorum}

\vspace{1em}
\gregorioscore{./mass/im-ego_dedi}
\ohstranslate{im-ego_dedi}

\vspace{1em}
\gregorioscore{./mass/im-ego_te_exaltavi}
\ohstranslate{im-ego_te_exaltavi}

%\separator{4}
\pagebreak

%%%%%%%%%%%%%%%%%%%%%%%%%%%%%%%%%%%%%%
\ohsformatresponsorium
\gresetinitiallines{1}
\ohsannotation{Hymn.}{I}
\gregorioscore{./mass/hy-crux_fidelis1}

\ohsannotation{}{I}
\gregorioscore{./mass/hy-crux_fidelis2}

\begin{small}\textit{Repetitur} Crux fidelis \textit{usque ad} * Dulce lignum.\end{small}
\gresetinitiallines{0}
\gregorioscore{./mass/hy-crux_fidelis3}
\vspace{1em}
\gregorioscore{./mass/hy-crux_fidelis4-hoc_opus}
\vspace{1em}
\gregorioscore{./mass/hy-crux_fidelis5-quando_venit}
\vspace{1em}
\gregorioscore{./mass/hy-crux_fidelis6-vagit_infans}
\vspace{1em}
\gregorioscore{./mass/hy-crux_fidelis7-lustris_sex}
\vspace{1em}
\gregorioscore{./mass/hy-crux_fidelis8-hic_acetum}
\vspace{1em}
\gregorioscore{./mass/hy-crux_fidelis9-flecte_ramos}
\vspace{1em}
\gregorioscore{./mass/hy-crux_fidelis10-sola_digna}
\begin{small}\textit{Conclúsio numquam omittenda.}\end{small}
\gregorioscore{./mass/hy-crux_fidelis11-gloria_et_honor}
%%%%%%%%%%%%%%%%%%%%%%%%%%%%%%%%%%%%%%

%\vspace{1em}
\begin{center}\greseparator{2}{20}\end{center}
%%%%%%%%%%%%%%%%%%%%%%%%%%%%%%%%%%%%%%%%%%%%%%%%%%%%%%%%%%%%%%%%%%%%%%%%%%%%%%%%%%%%%%%%%%%%%%%%
%%%%%%%%%%%%%%%%%%%%%%%%%%%%%%%%%%%%%%%%%%%%%%%%%%%%%%%%%%%%%%%%%%%%%%%%%%%%%%%%%%%%%%%%%%%%%%%%

