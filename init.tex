% !TEX TS-program = lualatex
% !TEX encoding = UTF-8

% This is a simple template for a LuaLaTeX document using gregorio scores.
%PATCH START - specify slovak in documentclass
%\documentclass[17pt,slovak]{extarticle} % use larger type; default would be 10pt
\documentclass[17pt,slovak,twoside,openright]{extreport} % use larger type; default would be 10pt
\raggedbottom

%PATCH END
%\usepackage[16pt]{extsizes}


\usepackage{titlesec}
 
\titleformat
{\chapter} % command
[display] % shape
{\bfseries\large\scshape\centering} % format
{\thechapter} % label
{0.5ex} % sep
{
%    \rule{\textwidth}{1pt}
	\vspace{-5ex}%
	\feria \\%
%
%    \centering
} % before-code
[
%\vspace{-0.5ex}%
%\rule{\textwidth}{0.3pt}
] % after-code

% usual packages loading:
\usepackage{luatextra}
\usepackage{graphicx} % support the \includegraphics command and options
\usepackage{geometry} % See geometry.pdf to learn the layout options. There are lots.
\geometry{a4paper,vmargin={20mm,20mm},hmargin={20mm,20mm},footskip={10mm}} % or letterpaper (US) or a5paper or....
\usepackage[autocompile]{gregoriotex}
%\usepackage{gregoriotex} % for gregorio score inclusion
%\usepackage{fullpage} % to reduce the margins
\grechangestaffsize{24}
% choose the language of the document here
\usepackage[slovak,latin]{babel}
\usepackage[nosingleletter]{impnattypo}
%\usepackage[slovak]{babel}
\usepackage{graphicx}

% use the two following package for using normal TeX fonts
\usepackage[T1]{fontenc}
\usepackage[utf8]{luainputenc}
% If you use usual TeX fonts, here is a starting point:
%\usepackage{times}
\usepackage{palatino}
%\usepackage{tgtermes} % (a free font)
% to change the font to something better, you can install the kpfonts package (if not already installed). To do so
% go open the "TeX Live Manager" in the Menu Start->All Programs->TeX Live 2010
%PATCH START
%\usepackage{parallel}
%\input{twocolumns-conf.tex}
%PATCH END

\setlength{\parindent}{0pt} 

%\def\gretextformat#1{%
%{\fontsize{17}{17}\selectfont #1\relax}%
%}

%\font \gregoriosymbolfont=gresym at 20pt

%\def\greabovelinestextstyle#1{%
%{\fontsize{12}{12}\selectfont #1\relax}%
%}
%\usepackage{parskip}
\setlength{\parskip}{2mm}
\usepackage{enumitem}
\usepackage{color}
\usepackage{fancyheadings}
\usepackage{lastpage}
\pagestyle{fancy}
% here we begin the document
%%%%%%%%%%%%%%%%%%%%%%%%%%%%%%%%%%%%%%%%%%%%%%%%%%%%%%%%%%
%%%%%%%%%%%%%%%%%%%%%%%%%%%%%%%%%%%%%%%%%%%%%%%%%%%%%%%%%%
\begin{document}
% The title:
%\begin{center}\begin{Large}\textsc{Veľký Piatok}\end{Large}\end{center}
% Here we set the space around the initial.
% Please report to http://home.gna.org/gregorio/gregoriotex/details for more details and options
%\setspaceafterinitial{2.2mm plus 0em minus 0em}
%\setspacebeforeinitial{2.2mm plus 0em minus 0em}

% Here we set the initial font. Change 43 if you want a bigger initial.
%\def\grechangestyle{initial}#1{%
%{\fontsize{43}{43}\selectfont #1}%
%}
%psalmodia
\vspace{6 mm}
\newcommand{\prizvuk}[1]{%
{\textbf{#1}}%
}
\newcommand{\priprava}[1]{%
{\textit{#1}}%
}
\newcommand{\flexa}[1]{%
{\underline{#1}}%
}
\newcommand{\nadpisa}[1]{%
{\begin{center}\large{\textbf{\textsc{#1}}}\end{center}}%
}
\newcommand{\nadpisb}[1]{%
{\begin{center}\large{\textsc{#1}}\end{center}}%
}
\newcommand{\nadpisc}[1]{%\usepackage{fancyheadings}
{\textit{\small{\textbf{#1}}}}%
}
\newcommand{\nadpisd}[1]{%
{\begin{center}\textit{\small{#1}}\end{center}}%
}
\newcommand{\nadpisleft}[1]{%
{\color{red} \textit{\small{#1}}}%
\vspace{0.2em}%
}
\newcommand{\nadpisleftclose}[1]{%
{\color{red} \textit{\small{#1}}}%
\vspace{-1em}%
}
\newcommand{\nadpissub}[2]{%
{\begin{center}\textbf{\textsc{#1}}
\linebreak
\textit{\small{#2}}\end{center}}%
}

%%%%%%%%%%%%%%%%%%%%%%%%%%%%%%%%%%%%%%%%%%%%%%%%%%%%%%%%%%
\newcommand{\textaccent}[1]{%
{\textbf{{#1}}}%
%{\color{red} \textbf{\large{#1}}}%
}
\newcommand{\textprep}[1]{%
%{\color{red} \textit{\large{#1}}}%
{\textit{{#1}}}%
}
\newcommand{\textflex}[1]{%
{\underline{#1}}%
}
%%%%%%%%%%%%%%%%%%%%%%%%%%%%%%%%%%%%%%%%%%%%%%%%%%%%%%%%%%
\newcommand{\separator}[1]{%
\nopagebreak[4]
\begin{center}\greseparator{3}{25}\end{center}
\pagebreak[{#1}]
}
%%%%%%%%%%%%%%%%%%%%%%%%%%%%%%%%%%%%%%%%%%%%%%%%%%%%%%%%%%
\newcommand{\oratio}[1]{%
\nadpisleft{Oratio:}

\textbf{\Vbar.} Dómine, exáudi oratiónem meam. \mbox{\textbf{\Rbar.} Et clamor meus ad te véniat.}



\vspace{1em}

\input{{#1}}

\vspace{1em}

\textbf{\Vbar.} Dómine, exáudi oratiónem meam. \mbox{\textbf{\Rbar.} Et clamor meus ad te véniat.}


}
%%%%%%%%%%%%%%%%%%%%%%%%%%%%%%%%%%%%%%%%%%%%%%%%%%%%%%%%%%
\newcommand{\ohschapter}[1]{%
\chapter*{{#1}}%
\thispagestyle{fancy}%
\def \ohschaptername {{#1}}
}
%%%%%%%%%%%%%%%%%%%%%%%%%%%%%%%%%%%%%%%%%%%%%%%%%%%%%%%%%%


\hyphenation{nech umu-če-ný}

\cfoot{\textit{\small{\thepage / \pageref{LastPage} - Officium Hebdomadæ sanctæ - \feriashort - \ohschaptername }}}


% We set red lines here, comment it if you want black ones.
%\redlines

% We set VIII above the initial.
%\greannotation{\small \textsc{\textbf{VIII}}}{\small \textsc{\textbf{VIII}}}

% We type a text in the top right corner of the score:
%\commentary{{\small \emph{Cf. Is. 30, 19 . 30 ; Ps. 79}}}

% and finally we include the score. The file must be in the same directory as this one.

%%%%%%%%%%%%%%%%%%%%%%%%%%%%%%%%%%%%%%%%%%%%%%%%%%%%%%%%%%%%%%%%%%%%%%%%%%%%%%%%%%%%%%%%%%%%%%%%
%%%%%%%%%%%%%%%%%%%%%%%%%%%%%%%%%%%%%%%%%%%%%%%%%%%%%%%%%%%%%%%%%%%%%%%%%%%%%%%%%%%%%%%%%%%%%%%%

