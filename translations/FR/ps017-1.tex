%\versseparator
Je t'aime, Seigneur, ma force : Seigneur, mon roc, ma forteresse,
\versseparator
Dieu mon libérateur, le rocher qui m'abrite, mon bouclier, mon fort, mon arme de victoire !
\versseparator
Louange à Dieu ! + Quand je fais appel au Seigneur, je suis sauvé de tous mes ennemis.
\versseparator
Les liens de la mort m'entouraient, le torrent fatal m'épouvantait ;
\versseparator
des liens infernaux m'étreignaient : j'étais pris aux pièges de la mort.
\versseparator
Dans mon angoisse, j'appelai le Seigneur ; vers mon Dieu, je lançai un cri ; de son temple il entend ma voix : mon cri parvient à ses oreilles.
\versseparator
La terre titube et tremble, + les assises des montagnes frémissent, secouées par l'explosion de sa colère.
\versseparator
Une fumée sort de ses narines, + de sa bouche, un feu qui dévore, une gerbe de charbons embrasés.
\versseparator
Il incline les cieux et descend, une sombre nuée sous ses pieds :
\versseparator
d'un kéroub, il fait sa monture, il vole sur les ailes du vent.
\versseparator
Il se cache au sein des ténèbres + et dans leurs replis se dérobe : nuées sur nuées, ténèbres diluviennes.
\versseparator
Une lueur le précède, + ses nuages déferlent : grêle et gerbes de feu.
\versseparator
Tonnerre du Seigneur dans le ciel, * le Très-Haut fait entendre sa voix : grêle et gerbes de feu.
\versseparator
De tous côtés, il tire des flèches, il décoche des éclairs, il répand la terreur.
\versseparator
Alors le fond des mers se découvrit, les assises du monde apparurent, sous ta voix menaçante, Seigneur, au souffle qu'exhalait ta colère.
