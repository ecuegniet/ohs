%\versseparator
Six jours avant la Pâque, Jésus vint à Béthanie où habitait Lazare, qu’il avait réveillé d’entre les morts.
On donna un repas en l’honneur de Jésus. Marthe faisait le service, Lazare était parmi les convives avec Jésus.
Or, Marie avait pris une livre d’un parfum très pur et de très grande valeur ; elle versa le parfum sur les pieds de Jésus, qu’elle essuya avec ses cheveux ; la maison fut remplie de l’odeur du parfum.
Judas Iscariote, l’un de ses disciples, celui qui allait le livrer, dit alors :
« Pourquoi n’a-t-on pas vendu ce parfum pour trois cents pièces d’argent, que l’on aurait données à des pauvres ? »
Il parla ainsi, non par souci des pauvres, mais parce que c’était un voleur : comme il tenait la bourse commune, il prenait ce que l’on y mettait.
Jésus lui dit : « Laisse-la observer cet usage en vue du jour de mon ensevelissement !
Des pauvres, vous en aurez toujours avec vous, mais moi, vous ne m’aurez pas toujours. »
Or, une grande foule de Juifs apprit que Jésus était là, et ils arrivèrent, non seulement à cause de Jésus, mais aussi pour voir ce Lazare qu’il avait réveillé d’entre les morts.
Les grands prêtres décidèrent alors de tuer aussi Lazare,
parce que beaucoup de Juifs, à cause de lui, s’en allaient, et croyaient en Jésus.
