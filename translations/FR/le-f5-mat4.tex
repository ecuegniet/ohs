%\versseparator
Exaucez ma prière, ô mon Dieu, et ne méprisez pas ma demande ; écoutez-moi, et exaucez-moi : ces paroles sont d’un homme plein de souci, vigilant, et plongé dans l’affliction. Etant accablé de plusieurs maux, il prie, souhaitant d’être délivré du mal.
Il nous reste de savoir l’espèce de son mal ; et quand il aura commencé de l’expliquer, nous reconnaîtrons que nous sommes dans le même état ; afin que nos peines communes nous engagent à unir nos prières. J’ai été attristé, dit-il, dans mes exercices, et j’ai été troublé. Où est-ce qu’il a été attristé ? Où est-ce qu’il a été troublé ?
C’est, dit-il, dans mes exercices. Il a désigné les méchants qui le font souffrir ; il donne le nom d’exercice à cette persécution des hommes méchants. Ne croyez pas que les méchants soient inutiles en ce monde, et que Dieu n’en retire aucun bien ; car tout méchant vit, ou pour se corriger, ou pour exercer la patience et la vertu des bons.