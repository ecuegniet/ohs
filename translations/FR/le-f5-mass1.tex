%\versseparator
En ces jours-là, dans le pays d’Égypte,
le Seigneur dit à Moïse et à son frère Aaron :
« Ce mois-ci
sera pour vous le premier des mois,
il marquera pour vous le commencement de l’année.
Parlez ainsi à toute la communauté d’Israël :
le dix de ce mois,
que l’on prenne un agneau par famille,
un agneau par maison.
Si la maisonnée est trop peu nombreuse pour un agneau,
elle le prendra avec son voisin le plus proche,
selon le nombre des personnes.
Vous choisirez l’agneau d’après ce que chacun peut manger.
Ce sera une bête sans défaut, un mâle, de l’année.
Vous prendrez un agneau ou un chevreau.
Vous le garderez jusqu’au quatorzième jour du mois.
Dans toute l’assemblée de la communauté d’Israël,
on l’immolera au coucher du soleil.
On prendra du sang,
que l’on mettra sur les deux montants et sur le linteau
des maisons où on le mangera.
On mangera sa chair cette nuit-là,
on la mangera rôtie au feu,
avec des pains sans levain et des herbes amères.
Vous mangerez ainsi : la ceinture aux reins,
les sandales aux pieds,
le bâton à la main.
Vous mangerez en toute hâte :
c’est la Pâque du Seigneur.
Je traverserai le pays d’Égypte, cette nuit-là ;
je frapperai tout premier-né au pays d’Égypte,
depuis les hommes jusqu’au bétail.
Contre tous les dieux de l’Égypte j’exercerai mes jugements :
Je suis le Seigneur.
Le sang sera pour vous un signe,
sur les maisons où vous serez.
Je verrai le sang, et je passerai :
vous ne serez pas atteints par le fléau
dont je frapperai le pays d’Égypte.
Ce jour-là
sera pour vous un mémorial.
Vous en ferez pour le Seigneur une fête de pèlerinage.
C’est un décret perpétuel : d’âge en âge vous la fêterez. »