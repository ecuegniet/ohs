%\versseparator
Écoutez-moi, îles lointaines ! Peuples éloignés, soyez attentifs ! J’étais encore dans le sein maternel quand le Seigneur m’a appelé ; j’étais encore dans les entrailles de ma mère quand il a prononcé mon nom.
Il a fait de ma bouche une épée tranchante, il m’a protégé par l’ombre de sa main ; il a fait de moi une flèche acérée, il m’a caché dans son carquois.
Il m’a dit : « Tu es mon serviteur, Israël, en toi je manifesterai ma splendeur. »
Et moi, je disais : « Je me suis fatigué pour rien, c’est pour le néant, c’est en pure perte que j’ai usé mes forces. » Et pourtant, mon droit subsistait auprès du Seigneur, ma récompense, auprès de mon Dieu.
Maintenant le Seigneur parle, lui qui m’a façonné dès le sein de ma mère pour que je sois son serviteur, que je lui ramène Jacob, que je lui rassemble Israël. Oui, j’ai de la valeur aux yeux du Seigneur, c’est mon Dieu qui est ma force.
Et il dit : « C’est trop peu que tu sois mon serviteur pour relever les tribus de Jacob, ramener les rescapés d’Israël : je fais de toi la lumière des nations, pour que mon salut parvienne jusqu’aux extrémités de la terre. »
