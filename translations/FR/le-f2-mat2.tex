Qui que tu sois, veux-tu être une âme fidèle, répands avec Marie sur les pieds du Seigneur un parfum précieux. Ce parfum, c’était la justice ; voilà pourquoi il pesait une livre ; c’était aussi un parfum « de nard » pur et précieux. Le nom de pisticus donné à ce parfum indique vraisemblablement la contrée d’où il venait, mais ce mot n’est pas mis sans dessein, et il est en parfait rapport avec le mystère dont il s’agit. Le mot grec pistis se rend en latin par fides, c’est-à-dire foi. Tu cherches à opérer la justice : « Le juste vit de la foi. » Oins les pieds de Jésus par une vie sainte, suis les traces du Seigneur. Essuie ses pieds avec tes cheveux ; si tu as du superflu, donne-le aux pauvres, et tu auras essuyé les pieds du Seigneur, car les cheveux semblent pour le corps quelque chose de superflu. Tu vois ce qu’il faut faire de ton superflu ; superflu pour toi, il est nécessaire aux pieds du Seigneur. Peut-être que, sur la terre, les pieds du Seigneur se trouvent dans le besoin.
