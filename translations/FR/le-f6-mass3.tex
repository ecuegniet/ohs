%\versseparator
Passion de Notre Seigneur Jésus-Christ selon saint Jean.
En ce temps-là : Jésus se rendit, accompagné de ses disciples, au-delà du torrent de Cédron, où il y avait un jardin, dans lequel il entra, lui et ses disciples. Judas, qui le trahissait, connaissait aussi ce lieu, parce que Jésus y était souvent allé avec ses disciples. Ayant donc pris la cohorte et des satellites fournis par les Pontifes et les Pharisiens, Judas y vint avec des lanternes, des torches et des armes. Alors Jésus, sachant tout ce qui devait lui arriver, s’avança et leur dit : "Qui cherchez-vous ?". Ils lui répondirent : "Jésus de Nazareth." Il leur dit : "C’est moi." Or, Judas, qui le trahissait, était là avec eux. Lors donc que Jésus leur eut dit : "C’est moi," ils reculèrent et tombèrent par terre. Il leur demanda encore une fois : "Qui cherchez-vous ?" Et ils dirent : "Jésus de Nazareth.". Jésus répondit : "Je vous l’ai dit, c’est moi. Si donc c’est moi que vous cherchez, laissez aller ceux-ci.". Il dit cela afin que fût accomplie la parole qu’il avait dite : "Je n’ai perdu aucun de ceux que vous m’avez donnés.". Alors, Simon-Pierre, qui avait une épée, la tira, et, frappant le serviteur du grand prêtre, il lui coupa l’oreille droite : ce serviteur s’appelait Malchus. Mais Jésus dit à Pierre : "Remets ton épée dans le fourreau. Ne boirai-je donc pas le calice que mon Père m’a donné ?". Alors la cohorte, le tribun et les satellites des Juifs se saisirent de Jésus et le lièrent. Ils l’emmenèrent d’abord chez Anne parce qu’il était beau-père de Caïphe, lequel était grand-prêtre cette année-là. Or, Caïphe était celui qui avait donné ce conseil aux Juifs : "Il est avantageux qu’un seul homme meure pour le peuple.". Cependant Simon-Pierre suivait Jésus avec un autre disciple. Ce disciple, étant connu du grand-prêtre, entra avec Jésus dans la cour du grand-prêtre,. Mais Pierre était resté près de la porte, en dehors. L’autre disciple, qui était connu du grand-prêtre sortit donc, parla à la portière, et fit entrer Pierre. Cette servante, qui gardait la porte, dit à Pierre : "N’es-tu pas, toi aussi, des disciples de cet homme ?" Il dit : "Je n’en suis point.". Les serviteurs et les satellites étaient rangés autour d’un brasier, parce qu’il faisait froid, et ils se chauffaient. Pierre se tenait aussi avec eux, et se chauffait. Le grand-prêtre interrogea Jésus sur ses disciples et sur sa doctrine. Jésus lui répondit : "J’ai parlé ouvertement au monde ; j’ai toujours enseigné dans la synagogue et dans le temple, où tous les Juifs s’assemblent, et je n’ai rien dit en secret. Pourquoi m’interroges-tu ? Demande à ceux qui m’ont entendu, ce que je leur ai dit ; eux ils savent ce que j’ai enseigné.". A ces mots, un des satellites qui se trouvait là, donna un soufflet à Jésus, en disant : "Est-ce ainsi que tu réponds au grand-prêtre ?". Jésus lui répondit : "Si j’ai mal parlé, fais voir ce que j’ai dit de mal ; mais si j’ai bien parlé, pourquoi me frappes-tu ?". Anne avait envoyé Jésus lié à Caïphe, le grand-prêtre. Or, Simon-Pierre était là, se chauffant. Ils lui dirent : "N’es-tu pas, toi aussi, de ses disciples ?". Un des serviteurs du grand-prêtre, parent de celui à qui Pierre avait coupé l’oreille, lui dit : "Ne t’ai-je pas vu avec lui dans le jardin ?". Pierre nia de nouveau et aussitôt le coq chanta. Ils conduisirent Jésus de chez Caïphe au prétoire : c’était le matin. Mais ils n’entrèrent pas eux-mêmes dans le prétoire, pour ne pas se souiller et afin de pouvoir manger la Pâque. Pilate sortit donc vers eux, et dit : "Quelle accusation portez-vous contre cet homme ?". Ils lui répondirent : "Si ce n’était pas un malfaiteur, nous ne te l’aurions pas livré.". Pilate leur dit : "Prenez-le vous-mêmes, et jugez-le selon votre loi." Les Juifs lui répondirent : "Il ne nous est pas permis de mettre personne à mort." :. Afin que s’accomplît la parole que Jésus avait dite, lorsqu’il avait indiqué de quelle mort il devait mourir. Pilate donc, étant rentré dans le prétoire, appela Jésus, et lui dit : "Es-tu le roi des Juifs ?". Jésus répondit : "Dis-tu cela de toi-même, ou d’autres te l’ont-ils dit de moi ?". Pilate répondit : "Est-ce que je suis Juif ? Ta nation et les chefs des prêtres t’ont livré à moi : qu’as-tu fait ?". Jésus répondit : "Mon royaume n’est pas de ce monde ; si mon royaume était de ce monde, mes serviteurs auraient combattu pour que je ne fusse pas livré aux Juifs, mais maintenant mon royaume n’est point d’ici-bas.". Pilate lui dit : "Tu es donc roi ?" Jésus répondit : "Tu le dis, je suis roi. Je suis né et je suis venu dans le monde pour rendre témoignage à la vérité : quiconque est de la vérité écoute ma voix.". Pilate lui dit : "Qu’est-ce que la vérité ?" Ayant dit cela, il sortit de nouveau pour aller vers les Juifs, et il leur dit :. "Pour moi, je ne trouve aucun crime en lui. Mais c’est la coutume qu’à la fête de Pâque je vous délivre quelqu’un. Voulez-vous que je vous délivre le roi des Juifs ?". Alors tous crièrent de nouveau : "Non, pas lui, mais Barabbas." Or, Barabbas était un brigand. Alors Pilate prit Jésus et le fit flageller. Et les soldats ayant tressé une couronne d’épines, la mirent sur sa tête, et le revêtirent d’un manteau de pourpre ;. Puis s’approchant de lui, ils disaient : "Salut, roi des Juifs !" et ils le souffletaient. Pilate sortit encore une fois et dit aux Juifs : "Voici que je vous l’amène dehors, afin que vous sachiez que je ne trouve en lui aucun crime.". Jésus sortit donc, portant la couronne d’épines et le manteau d’écarlate, et Pilate leur dit : "Voici l’homme.". Lorsque les Princes des prêtres et les satellites le virent, ils s’écrièrent : "Crucifie-le, crucifie-le !" Pilate leur dit : "Prenez-le vous-mêmes, et crucifiez-le ; car pour moi, je ne trouve aucun crime en lui.". Les Juifs lui répondirent : "Nous avons une loi, et d’après notre loi, il doit mourir, parce qu’il s’est fait Fils de Dieu". Ayant entendu ces paroles, Pilate fut encore plus effrayé. Et rentrant dans le prétoire, il dit à Jésus : "D’où es-tu ?" Mais Jésus ne lui fit aucune réponse. Pilate lui dit : "C’est à moi que tu ne parles pas ? Ignores-tu que j’ai le pouvoir de te délivrer et le pouvoir de te crucifier ?". Jésus répondit : "Tu n’aurais sur moi aucun pouvoir, s’il ne t’avait pas été donné d’en haut. C’est pourquoi celui qui m’a livré à toi a un plus grand péché.". Dès ce moment, Pilate cherchait à le délivrer. Mais les Juifs criaient, disant : "Si tu le délivres, tu n’es point ami de César ; quiconque se fait roi, se déclare contre César.". Pilate, ayant entendu ces paroles, fit conduire Jésus dehors, et il s’assit sur son tribunal, au lieu appelé Lithostrotos, et en hébreu Gabbatha. C’était la Préparation de la Pâque, et environ la sixième heure. Pilate dit aux Juifs : "Voici votre roi.". Mais ils se mirent à crier : "Qu’il meure ! Qu’il meure ! Crucifie-le." Pilate leur dit : "Crucifierai-je votre roi ?" les Princes des prêtres répondirent : "Nous n’avons de roi que César.". Alors il le leur livra pour être crucifié. Et ils prirent Jésus et l’emmenèrent. Jésus, portant sa croix, arriva hors de la ville au lieu nommé Calvaire, en Hébreu Golgotha ;. C’est là qu’ils le crucifièrent, et deux autres avec lui, un de chaque côté, et Jésus au milieu. Pilate fit aussi une inscription, et la fit mettre au haut de la croix ; Elle portait ces mots : "Jésus de Nazareth, le roi des Juifs.". Beaucoup de Juifs lurent cet écriteau, car le lieu où Jésus avait été crucifié était près de la ville, et l’inscription était en hébreu, en grec et en latin. Or les princes des prêtres des Juifs dirent à Pilate : "Ne mets pas : Le roi des Juifs, mais que lui-même a dit : Je suis le roi des Juifs.". Pilate répondit : "Ce que j’ai écrit, je l’ai écrit.". Les soldats, après avoir crucifié Jésus, prirent ses vêtements, et ils en firent quatre parts, une pour chacun d’eux. Ils prirent aussi sa tunique : c’était une tunique sans couture, d’un seul tissu depuis le haut jusqu’en bas. Ils se dirent donc entre eux : "Ne la déchirons pas, mais tirons au sort à qui elle sera." ; afin que s’accomplît cette parole de l’Écriture : "Ils se sont partagé mes vêtements, et ils ont tiré ma robe au sort." C’est ce que firent les soldats. Près de la croix de Jésus se tenaient sa mère et la sœur de sa mère, Marie, femme de Clopas, et Marie-Madeleine. Jésus ayant vu sa mère, et auprès d’elle le disciple qu’il aimait, dit à sa mère : "Femme, voilà votre fils.". Ensuite il dit au disciple : "Voilà votre mère." Et depuis cette heure-là, le disciple la prit chez lui. Après cela, Jésus sachant que tout était maintenant consommé, afin que l’Écriture s’accomplît, dit : "J’ai soif.". Il y avait là un vase plein de vinaigre ; les soldats en remplirent une éponge, et l’ayant fixée au bout d’une tige d’hysope, ils l’approchèrent de sa bouche. Quand Jésus eut pris le vinaigre, il dit : "Tout est consommé", et baissant la tête il rendit l’esprit.
Or, comme c’était la Préparation, de peur que les corps ne restassent sur la croix pendant le sabbat, car le jour de ce sabbat était très solennel, les Juifs demandèrent à Pilate qu’on rompît les jambes aux crucifiés et qu’on les détachât. Les soldats vinrent donc, et ils rompirent les jambes du premier, puis de l’autre qui avait été crucifié avec lui. Mais quand ils vinrent à Jésus, le voyant déjà mort, ils ne lui rompirent pas les jambes. Mais un des soldats lui transperça le côté avec sa lance, et aussitôt il en sortit du sang et de l’eau. Et celui qui l’a vu en rend témoignage, et son témoignage est vrai ; et il sait qu’il dit vrai, afin que vous aussi vous croyiez. Car ces choses sont arrivées afin que l’Écriture fut accomplie : "Aucun de ses os ne sera rompu.". Et il est encore écrit ailleurs : "Ils regarderont celui qu’ils ont transpercé." Après cela, Joseph d’Arimathie, qui était disciple de Jésus, mais en secret par crainte des Juifs, demanda à Pilate d’enlever le corps de Jésus. Et Pilate le permit. Il vint donc, et prit le corps de Jésus. Nicodème, qui était venu la première fois trouver Jésus de nuit, vint aussi, apportant un mélange de myrrhe et d’aloès, d’environ cent livres. Ils prirent donc le corps de Jésus, et l’enveloppèrent dans des linges, avec les aromates, selon la manière d’ensevelir en usage chez les Juifs. Or, au lieu où Jésus avait été crucifié, il y avait un jardin, et dans le jardin un sépulcre neuf, où personne n’avait encore été mis. C’est là, à cause de la Préparation des Juifs, qu’ils déposèrent Jésus, parce que le sépulcre était proche.