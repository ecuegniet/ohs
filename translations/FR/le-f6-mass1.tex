%\versseparator
Mon serviteur réussira, dit le Seigneur ;
il montera, il s’élèvera, il sera exalté !
La multitude avait été consternée en le voyant,
car il était si défiguré
qu’il ne ressemblait plus à un homme ;
il n’avait plus l’apparence d’un fils d’homme.
Il étonnera de même une multitude de nations ;
devant lui les rois resteront bouche bée,
car ils verront ce que, jamais, on ne leur avait dit,
ils découvriront ce dont ils n’avaient jamais entendu parler.
\versseparator
Qui aurait cru ce que nous avons entendu ?
Le bras puissant du Seigneur, à qui s’est-il révélé ?
Devant lui, le serviteur a poussé comme une plante chétive,
une racine dans une terre aride ;
il était sans apparence ni beauté qui attire nos regards,
son aspect n’avait rien pour nous plaire.
Méprisé, abandonné des hommes,
homme de douleurs, familier de la souffrance,
il était pareil à celui devant qui on se voile la face ;
et nous l’avons méprisé, compté pour rien.
En fait, c’étaient nos souffrances qu’il portait,
nos douleurs dont il était chargé.
Et nous, nous pensions qu’il était frappé,
meurtri par Dieu, humilié.
Or, c’est à cause de nos révoltes qu’il a été transpercé,
à cause de nos fautes qu’il a été broyé.
Le châtiment qui nous donne la paix a pesé sur lui :
par ses blessures, nous sommes guéris.
Nous étions tous errants comme des brebis,
chacun suivait son propre chemin.
Mais le Seigneur a fait retomber sur lui
nos fautes à nous tous.
\versseparator
Maltraité, il s’humilie,
il n’ouvre pas la bouche :
comme un agneau conduit à l’abattoir,
comme une brebis muette devant les tondeurs,
il n’ouvre pas la bouche.
Arrêté, puis jugé, il a été supprimé.
Qui donc s’est inquiété de son sort ?
Il a été retranché de la terre des vivants,
frappé à mort pour les révoltes de son peuple.
On a placé sa tombe avec les méchants,
son tombeau avec les riches ;
et pourtant il n’avait pas commis de violence,
on ne trouvait pas de tromperie dans sa bouche.
Broyé par la souffrance, il a plu au Seigneur.
S’il remet sa vie en sacrifice de réparation,
il verra une descendance, il prolongera ses jours :
par lui, ce qui plaît au Seigneur réussira.
\versseparator
Par suite de ses tourments, il verra la lumière,
la connaissance le comblera.
Le juste, mon serviteur, justifiera les multitudes,
il se chargera de leurs fautes.
C’est pourquoi, parmi les grands, je lui donnerai sa part,
avec les puissants il partagera le butin,
car il s’est dépouillé lui-même
jusqu’à la mort,
et il a été compté avec les pécheurs,
alors qu’il portait le péché des multitudes
et qu’il intercédait pour les pécheurs.