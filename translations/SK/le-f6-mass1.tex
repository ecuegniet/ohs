%\versseparator
Hľa, úspech bude mať môj služobník, pozdvihne, vyvýši, zvelebí sa veľmi.
Ako sa nad ním zhrozili mnohí, – veď neľudsky je znetvorený jeho výzor a jeho obraz je nepodobný človeku – tak ho budú obdivovať mnohé národy, králi si pred ním zatvoria ústa.
Veď uvidia, o čom sa im nevravelo, a poznajú, čo neslýchali! 
Kto by uveril, čo sme počuli, a komu sa zjavilo Pánovo rameno?
\versseparator
Veď vzišiel pred ním sťa ratoliestka a ako koreň z vyschnutej zeme.
Nemá podoby ani krásy, aby sme hľadeli na neho, a nemá výzoru, aby sme po ňom túžili. Opovrhnutý a posledný z ľudí, muž bolestí, ktorý poznal utrpenie, pred akým si zakrývajú tvár, opovrhnutý, a preto sme si ho nevážili.
\versseparator
Vskutku on niesol naše choroby a našimi bôľmi sa on obťažil, no my sme ho pokladali za zbitého, strestaného Bohom a pokoreného.
On však bol prebodnutý pre naše hriechy, strýznený pre naše neprávosti, na ňom je trest pre naše blaho a jeho ranami sme uzdravení.
\versseparator
Všetci sme blúdili ako ovce, išli sme každý vlastnou cestou; a Pán na neho uvalil neprávosť nás všetkých. 
Obetoval sa, pretože sám chcel, a neotvoril ústa; ako baránka viedli ho na zabitie a ako ovcu, čo onemie pred svojím strihačom, (a neotvoril ústa).
\versseparator
Z úzkosti a súdu ho vyrvali a kto pomyslí na jeho pokolenie?
Veď bol vyťatý z krajiny živých, pre hriech svojho ľudu dostal úder smrteľný.
So zločincami mu dali hrob, jednako s boháčom bol v smrti, lebo nerobil násilie, ani podvod nemal v ústach.
\versseparator
Pánovi sa však páčilo zdrviť ho utrpením… Ak dá svoj život na obetu za hriech, uvidí dlhoveké potomstvo a podarí sa skrze neho vôľa Pánova.
Po útrapách sa jeho duša nahľadí dosýta. Môj spravodlivý služobník svojou vedomosťou ospravedlní mnohých a on ponesie ich hriechy.
\versseparator
Preto mu dám za údel mnohých a početných dostane za korisť, lebo vylial svoju dušu na smrť a pripočítali ho k hriešnikom.
On však niesol hriechy mnohých a prihováral sa za zločincov.