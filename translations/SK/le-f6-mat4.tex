Bráň ma pred zberbou ničomníkov 
pred tlupou zločincov. (Žalm 64, verš 2)
Uprime teraz náš pohľad na našu hlavu.
Mnoho mučeníkov trpelo, ale nie tak ako najväčší z nich.
A práve na ňom vidíme, ako sa im to podarilo.
Bol ochraňovaný od zberby ničomníkov, ochraňovaný Bohom.
Ochraňuje seba samého vo svojom Synovi a človeka, ktorý ho prosí; pretože je synom človeka a synom Božím. Je synom Božím, pretože má prirodzenosť Boha, je synom človeka, pretože má prirodzenosť sluhu (Filipanom 2;6-7). 
V jeho moci bolo ponechať si život, ale on ho dal dobrovoľne.
Čo mu mohli spraviť jeho nepriatelia?
Mohli zabiť jeho telo, nie však jeho dušu. Dobre to zvážte.
Bolo by nedostatočné, ak by Pán iba hovoril o mučeníctve, ak by toto slovo neposilnil príkladom.