%\versseparator
Ježiš vyšiel so svojimi učeníkmi za potok Cedron. Tam bola záhrada. Vošiel do nej on i jeho učeníci. 
O tom mieste však vedel aj jeho zradca Judáš, lebo Ježiš sa tam často schádzal so svojimi učeníkmi. Judáš vzal kohortu a sluhov od veľkňazov a farizejov a prišiel ta s lampášmi, fakľami a zbraňami. Ale Ježiš, keďže vedel všetko, čo malo naňho prísť, popodišiel a opýtal sa ich: „Koho hľadáte?“ 
Odpovedali mu: „Ježiša Nazaretského.“ Povedal im: „Ja som.“ Bol s nimi aj zradca Judáš. Ako im povedal: „Ja som,“ cúvli a popadali na zem. Znova sa ich teda opýtal: „Koho hľadáte?“ Oni povedali: „Ježiša Nazaretského.“ Ježiš odvetil: „Povedal som vám: Ja som. Keď teda mňa hľadáte, týchto nechajte odísť!“ Tak sa malo splniť slovo, ktoré povedal: „Z tých, ktorých si mi dal, nestratil som ani jedného.“ Šimon Peter mal meč. Vytasil ho, zasiahol ním veľkňazovho sluhu a odťal mu pravé ucho. Sluha sa volal Malchus. Ale Ježiš Petrovi povedal: „Schovaj meč do pošvy! Azda nemám piť kalich, ktorý mi dal Otec?!“
\versseparator
Kohorta, veliteľ a židovskí sluhovia Ježiša chytili, zviazali ho 
a priviedli najprv k Annášovi; bol totiž tesťom Kajfáša, ktorý bol veľkňazom toho roka. A bol to Kajfáš, čo poradil Židom: „Je lepšie, ak zomrie jeden človek za ľud.“
Za Ježišom šiel Šimon Peter a iný učeník. Ten učeník sa poznal s veľkňazom a vošiel s Ježišom do veľkňazovho dvora, Peter však ostal vonku pri dverách. Potom ten druhý učeník, čo sa poznal s veľkňazom, vyšiel, prehovoril s vrátničkou a voviedol ta Petra. Tu vrátnička povedala Petrovi: „Nie si aj ty z učeníkov toho človeka?“ On vravel: „Nie som.“ Stáli tam sluhovia a strážnici, ktorí si rozložili oheň, lebo bolo chladno, a zohrievali sa. S nimi stál aj Peter a zohrieval sa.
Veľkňaz sa vypytoval Ježiša na jeho učeníkov a na jeho učenie. Ježiš mu odpovedal: „Ja som verejne hovoril svetu. Vždy som učil v synagóge a v chráme, kde sa schádzajú všetci Židia, a nič som nehovoril tajne. Prečo sa pýtaš mňa? Opýtaj sa tých, ktorí počuli, čo som im hovoril! Oni vedia, čo som hovoril.“ Ako to povedal, jeden zo sluhov, čo tam stál, udrel Ježiša po tvári a povedal: „Tak odpovedáš veľkňazovi?“ Ježiš mu odvetil: „Ak som zle povedal, dokáž, čo bolo zlé, ale ak dobre, prečo ma biješ?!“ A tak ho Annáš zviazaného poslal k veľkňazovi Kajfášovi.
Šimon Peter tam stál a zohrieval sa. I pýtali sa ho: „Nie si aj ty z jeho učeníkov?“ On zaprel: „Nie som.“ Jeden z veľkňazových sluhov, príbuzný toho, ktorému Peter odťal ucho, vravel: „A nevidel som ťa s ním v záhrade?!“ Peter znova zaprel – a vtom zaspieval kohút.
\versseparator
Od Kajfáša viedli Ježiša do vládnej budovy. Bolo už ráno. Ale oni do vládnej budovy nevošli, aby sa nepoškvrnili a mohli jesť veľkonočného baránka. Preto vyšiel von za nimi Pilát a opýtal sa: „Akú žalobu podávate proti tomuto človeku?“ Odpovedali mu: „Keby tento nebol zločinec, neboli by sme ti ho vydali.“ Pilát im povedal: „Vezmite si ho vy a súďte podľa svojho zákona!“ Židia mu odpovedali: „My nesmieme nikoho usmrtiť.“ 
Tak sa malo splniť Ježišovo slovo, ktorým naznačil, akou smrťou zomrie.
Pilát opäť vošiel do vládnej budovy. Predvolal si Ježiša a spýtal sa ho: „Si židovský kráľ?“ Ježiš odpovedal: „Hovoríš to sám od seba, alebo ti to iní povedali o mne?“ Pilát odvetil: „Vari som ja Žid? Tvoj národ a veľkňazi mi ťa vydali. Čo si vykonal?“ Ježiš povedal: „Moje kráľovstvo nie je z tohto sveta. Keby moje kráľovstvo bolo z tohto sveta, moji služobníci by sa bili, aby som nebol vydaný Židom. Lenže moje kráľovstvo nie je stadiaľto.“ Pilát mu povedal: „Tak predsa si kráľ?“ Ježiš odpovedal: „Sám hovoríš, že som kráľ. Ja som sa na to narodil a na to som prišiel na svet, aby som vydal svedectvo pravde. Každý, kto je z pravdy, počúva môj hlas.“ Pilát mu povedal: „Čo je pravda?“ Ako to povedal, znova vyšiel k Židom a vravel im: „Ja na ňom nenachádzam nijakú vinu. Je však u vás zvykom, že vám na Veľkú noc prepúšťam jedného väzňa. Chcete teda, aby som vám prepustil židovského kráľa?“ Oni znova kričali: „Toho nie, ale Barabáša!“ A Barabáš bol zbojník.
\versseparator
Vtedy Pilát Ježiša vzal a dal ho zbičovať. Vojaci uplietli z tŕnia korunu, položili mu ju na hlavu a odeli ho do purpurového plášťa. Prichádzali k nemu a hovorili: „Buď pozdravený, židovský kráľ!“ A bili ho po tvári. Pilát znova vyšiel a povedal im: „Pozrite, privádzam vám ho von, aby ste vedeli, že na ňom nijakú vinu nenachádzam.“ Ježiš vyšiel von s tŕňovou korunou a v purpurovom plášti. Pilát im povedal: „Hľa, človek!“ Len čo ho zazreli veľkňazi a ich sluhovia, kričali: „Ukrižuj! Ukrižuj ho!“ Pilát im povedal: „Vezmite si ho a ukrižujte. Ja na ňom nenachádzam vinu.“ 
Židia mu odpovedali: „My máme zákon a podľa zákona musí umrieť, lebo sa vydával za Božieho Syna.“
Keď to Pilát počul, ešte väčšmi sa naľakal. Znova vošiel do vládnej budovy a spýtal sa Ježiša: „Odkiaľ si?“ Ale Ježiš mu neodpovedal. Pilát sa ho spýtal: „So mnou sa nechceš rozprávať?! Nevieš, že mám moc prepustiť ťa a moc ukrižovať ťa?“ 
Ježiš odpovedal: „Nemal by si nado mnou nijakú moc, keby ti to nebolo dané zhora. Preto má väčší hriech ten, čo ma vydal tebe.“
\versseparator
Od tej chvíle sa Pilát usiloval prepustiť ho. Ale Židia kričali: „Ak ho prepustíš, nie si priateľom cisára. Každý, kto sa vydáva za kráľa, stavia sa proti cisárovi.“ Keď Pilát počul tieto slová, vyviedol Ježiša von a sadol si na súdnu stolicu na mieste zvanom Lithostrotus, po hebrejsky Gabbatha. Bol Prípravný deň pred Veľkou nocou, okolo poludnia. Tu povedal Židom: „Hľa, váš kráľ!“ Ale oni kričali: „Preč s ním! Preč s ním! Ukrižuj ho!“ Pilát im povedal: „Vášho kráľa mám ukrižovať?!“ Veľkňazi odpovedali: „Nemáme kráľa, iba cisára!“ 
Tak im ho teda vydal, aby ho ukrižovali. A oni prevzali Ježiša.
\versseparator
Sám si niesol kríž a vyšiel na miesto, ktoré sa volá Lebka, po hebrejsky Golgota. Tam ho ukrižovali a s ním iných dvoch, z jednej i druhej strany, Ježiša v prostriedku. Pilát vyhotovil aj nápis a pripevnil ho na kríž. Bolo tam napísané: „Ježiš Nazaretský, židovský kráľ.“ Tento nápis čítalo mnoho Židov, lebo miesto, kde Ježiša ukrižovali, bolo blízko mesta; a bol napísaný po hebrejsky, latinsky a grécky. Židovskí veľkňazi povedali Pilátovi: „Nepíš: Židovský kráľ, ale: On povedal: »Som židovský kráľ.«“ Pilát odpovedal: „Čo som napísal, to som napísal.“
\versseparator
Keď vojaci Ježiša ukrižovali, vzali jeho šaty a rozdelili ich na štyri časti, pre každého vojaka jednu. Vzali aj spodný odev. Ale tento odev bol nezošívaný, odhora v celku utkaný. 
Preto si medzi sebou povedali: „Netrhajme ho, ale losujme oň, čí bude!“ Aby sa splnilo Písmo:
„Rozdelili si moje šaty
a o môj odev hodili lós.“
A vojaci to tak urobili.
\versseparator
Pri Ježišovom kríži stála jeho matka, sestra jeho matky, Mária Kleopasova, a Mária Magdaléna. Keď Ježiš uzrel matku a pri nej učeníka, ktorého miloval, povedal matke: „Žena, hľa, tvoj syn!“ Potom povedal učeníkovi: „Hľa, tvoja matka!“ A od tej hodiny si ju učeník vzal k sebe.
\versseparator
Potom Ježiš vo vedomí, že je už všetko dokonané, povedal, aby sa splnilo Písmo: „Žíznim.“ 
Bola tam nádoba plná octu. Nastokli teda na yzop špongiu naplnenú octom a podali mu ju k ústam. Keď Ježiš okúsil ocot, povedal: „Je dokonané.“ Naklonil hlavu a odovzdal ducha.
\versseparator
Keďže bol Prípravný deň, Židia požiadali Piláta, aby ukrižovaným polámali nohy a sňali ich, aby nezostali telá na kríži cez sobotu, lebo v tú sobotu bol veľký sviatok. Prišli teda vojaci a polámali kosti prvému aj druhému, čo boli s ním ukrižovaní. No keď prišli k Ježišovi a videli, že je už mŕtvy, kosti mu nepolámali, 
ale jeden z vojakov mu kopijou prebodol bok a hneď vyšla krv a voda. A ten, ktorý to videl, vydal o tom svedectvo a jeho svedectvo je pravdivé. On vie, že hovorí pravdu, aby ste aj vy uverili. 
Toto sa stalo, aby sa splnilo Písmo: „Kosť mu nebude zlomená.“ 
A na inom mieste Písmo hovorí: „Uvidia, koho prebodli.“
\versseparator
Potom Jozef z Arimatey, ktorý bol Ježišovým učeníkom, ale tajným, lebo sa bál Židov, poprosil Piláta, aby mu dovolil sňať Ježišovo telo. A Pilát dovolil. Išiel teda a sňal jeho telo. Prišiel aj Nikodém, ten, čo bol kedysi u neho v noci. Priniesol asi sto libier zmesi myrhy s aloou. Vzali Ježišovo telo a zavinuli ho do plátna s voňavými olejmi, ako je u Židov zvykom pochovávať. V tých miestach, kde bol ukrižovaný, bola záhrada a v záhrade nový hrob, v ktorom ešte nik neležal. 
Tam teda uložili Ježiša, lebo bol židovský Prípravný deň a hrob bol blízko.
