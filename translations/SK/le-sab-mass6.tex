%\versseparator
Počuj, Izrael, príkazy života, zachyť uchom, nauč sa múdrosti!
Čo je to, Izrael? Čo, že si v nepriateľskej krajine?
Zostarel si v cudzej krajine, poškvrnený si mŕtvolami, prirátaný si k tým, čo klesli do podsvetia.
Opustil si prameň múdrosti.
Keby si bol kráčal cestou Boha, bol by si býval vo večnom pokoji.
Nauč sa, kde je múdrosť, kde je sila, kde je rozumnosť, tak poznáš zároveň, kde je dlhý život, kde žitie, kde je svetlo očí a pokoj.
Ktože našiel jej miesto a kto vkročil do jej pokladníc?
\versseparator
Ten však, čo všetko vie, pozná ju, objavil ju svojou rozumnosťou; ten, čo pripravil pre večné časy zem a naplnil ju štvornohými zvieratami.
Ak vyšle svetlo, ono ide, ak ho zavolá, poslúchne ho s chvením.
A hviezdy svietia na svojich strážach a tešia sa; ak ich zavolá, odpovedia: „Tu sme!“ a veselo svietia tomu, čo ich učinil.
\versseparator
Toto je náš Boh a iného popri ňom uznať nemožno.
Objavil každú cestu múdrosti a dal ju svojmu sluhovi Jakubovi a svojmu miláčikovi Izraelovi.
Potom sa zjavila na zemi a žila s ľuďmi.
Ona je knihou Božích príkazov a zákonom, ktorý trvá naveky: všetci, čo sa jej chopia, dosiahnu život, čo ju však opustia, zomrú.
\versseparator
Obráť sa, Jakub, a chyť sa jej, choď k jej svetlu, v ústrety jej žiare.
Neprepusť svoju slávu inému, ani čo cenného máš cudziemu národu.
\versseparator
Blahoslavení sme, Izrael, lebo nám je známe, čo sa ľúbi Bohu.