\nadpisleft{Lectio 4 (In Psalmum 54. ad 1. versum)}

Ex tractátu sancti Augustíni Epíscopi super Psalmos

Exáudi, Deus, oratiónem meam, et ne despéxeris deprecatiónem meam: inténde mihi, 
et exáudi me. Satagéntis, solliciti, in tribulatióne pósiti, verba sunt ista. 
Orat multa pátiens, de malo liberari desiderans. 
Superest ut videámus in quo malo sit: et cum dicere cœperit, 
agnoscámus ibi nos esse: ut communicáta tribulatióne, conjungámus oratiónem. 
Contristátus sum, inquit, in exercitatióne mea, et conturbátus sum. 
Ubi contristátus? ubi conturbátus? In exercitatióne mea, inquit. 
Hómines malos, quos pátitur, commemorátus est: 
eamdemque passiónem malórum hóminum exercitatiónem suam dixit. 
Ne putetis gratis esse malos in hoc mundo, et nihil boni de illis ágere Deum. 
Omnis malus aut ideo vivit, ut corrigátur; aut ideo vivit, ut per illum bonus exerceátur.
\par