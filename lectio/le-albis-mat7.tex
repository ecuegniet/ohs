\begin{small}
\textbf{\Vbar.} Jube, domne, benedícere.
\nadpisleft{Benedictio:} 
Evangélica léctio sit nobis salus et protéctio. \textbf{\Rbar.} Amen.
\end{small}

\nadpisleft{Lectio 7 (Joannes 20:19-31; Homilia 26 in Evangelia)}

Léctio sancti Evangélii secúndum Joánnem

In illo témpore: 
Cum sero esset die illo, una sabbatórum, et fores essent clausæ, 
ubi erant discípuli congregáti propter metum Judæórum: 
venit Jesus, et stetit in médio, et dixit eis: Pax vobis. 

Et réliqua.

Homilía sancti Gregórii Papæ.

Prima lectiónis hujus evangélicæ qu\'{\ae}stio ánimum pulsat: \\
quómodo post resurrectiónem corpus Domínicum verum fuit, \\
quod clausis jánuis ad discípulos íngredi pótuit? 

Sed sciéndum nobis est, quod divína operátio, si ratióne comprehénditur, non est admirábilis: \\
nec fides habet méritum, cui humána rátio præbet experiméntum. 

Sed hæc ipsa nostri Redemptóris opera, quæ ex semetípsis comprehéndi nequáquam possunt, 
ex ália ejus operatióne pensánda sunt: 
ut rebus mirabílibus fidem pr\'{\ae}beant facta mirabíliora. 

Illud enim corpus Dómini intrávit ad discípulos jánuis clausis, \\
quod vidélicet ad humános óculos per nativitátem suam clauso exívit útero Vírginis. 

Quid ergo mirum, si clausis jánuis post resurrectiónem suam in ætérnum jam victúrus intrávit, \\
qui moritúrus véniens, non apérto útero Vírginis exívit?

\textbf{\Vbar.} Tu autem, Dómine, miserére nobis.
\textbf{\Rbar.} Deo grátias.

