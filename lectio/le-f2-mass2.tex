\nadpisleft{Evangelium (Io 12, 1-11)}

\textbf{\Vbar.} Dóminus vobíscum.
\textbf{\Rbar.} Et cum spíritu tuo.

\textbf{\Vbar.} Lectio sancti Evapgelii secundum Joannem.
\textbf{\Rbar.} Glória tibi, Dómine.

Ante sex dies Paschae venit Bethániam, ubi erat Lázarus, quem suscitávit a mórtuis Iésus. Fecérunt ergo ei cenam ibi, et Martha ministrábat, Lázarus vero unus erat ex discumbéntibus cum eo. María ergo accépit libram unguénti nardi puri, pretiósi, et unxit pedes Iésu et extérsit capíllis suis pedes eíus; domus autem impléta est ex odóre unguénti. Dicit autem Iúdas Iscariótes, unus ex discípulis eíus, qui erat eum traditúrus: “quáre hoc unguéntum non véniit trecéntis denáriis et datum est egénis?” Dixit autem hoc, non quia de egénis pertinébat ad eum, sed quia fur erat et, lóculos habens, ea, quae mittebántur, portábat. Dixit ergo Iésus: “Sine illam, ut in diem sepultúrae meae servet illud. Páuperes enim semper habétis vobíscum, me autem non semper habétis.” Cognóvit ergo turba multa ex Iudǽis quia illic est, et venérunt non propter Iésum tantum, sed ut et Lázarum vidérent, quem suscitávit a mórtuis. Cogitavérunt autem príncipes sacerdótum, ut et Lázarum interfícerent, quia multi propter illum abíbant ex Iudǽis et credébant in Iésum.

\textbf{\Vbar.} Verbum Dómini.
\textbf{\Rbar.} Laus tibi, Christe.
\par