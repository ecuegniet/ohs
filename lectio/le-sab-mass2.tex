\nadpisleft{Lectio 2 (Gen 22:1-18)}

Léctio libri Génesis

In diébus illis:
Tentávit Deus Abraham et dixit ad eum: «Abraham». Ille respóndit: «Adsum». Ait:
«Tolle fílium tuum unigénitum, quem díligis, Isaac et vade in terram Moría; atque offer eum ibi in holocáustum super unum móntium, quem monstrávero tibi».

Igitur Abraham de nocte consúrgens stravit ásinum suum ducens secum duos iúvenes suos et Isaac fílium suum. Cumque concidísset ligna in holocáustum, surréxit et ábiit ad locum, quem præcéperat ei Deus. 
Die autem tértio, elevátis óculis, vidit locum procul dixítque ad púeros suos: «Exspectáte hic cum ásino. Ego et puer illuc usque properántes, postquam adoravérimus, revertémur ad vos».

Tulit quoque ligna holocáusti et impósuit super Isaac fílium suum; ipse vero portábat in mánibus ignem et cultrum. Cumque duo pérgerent simul, dixit Isaac Abrahæ patri suo: «Pater mi». Ille respóndit: «Quid vis, fili?». «Ecce, inquit, ignis et ligna; ubi est víctima holocáusti?». Dixit Abraham: «Deus providébit sibi víctimam holocáusti, fili mi». Pergébant ambo páriter;

Et venérunt ad locum, quem osténderat ei Deus, in quo ædificávit Abraham altáre
et désuper ligna compósuit. Cumque colligásset Isaac fílium suum, pósuit eum in altári super struem lignórum extendítque Abraham manum et arrípuit cultrum, ut immoláret fílium suum. 
Et ecce ángelus Dómini de cælo clamávit: «Abraham, Abraham». Qui respóndit: «Adsum». Dixítque: «Non exténdas manum tuam super púerum neque fácias illi quidquam. Nunc cognóvi quod times Deum et non pepercísti fílio tuo unigénito propter me».
Levávit Abraham óculos suos vidítque aríetem unum inter vepres hæréntem córnibus; quem assúmens óbtulit holocáustum pro fílio. Appellavítque nomen loci illíus: «Dóminus videt». Unde usque hódie dícitur: «In monte Dóminus vidétur».

Vocávit autem ángelus Dómini Abraham secúndo de cælo et dixit: «Per memetípsum iurávi, dicit Dóminus: quia fecísti hanc rem et non pepercísti fílio tuo unigénito, benedícam tibi et multiplicábo semen tuum sicut stellas cæli et velut arénam, quæ est in lítore maris.
Possidébit semen tuum portas inimicórum suórum, et benedicéntur in sémine tuo omnes gentes terræ, quia obœdísti voci meæ».

\textbf{\Vbar.} Verbum Dómini.
\textbf{\Rbar.} Deo grátias.
\par