\nadpisleft{Lectio 1 (Ex 12, 1-8. 11-14)}

Léctio libri Exodi

In diébus illis:
Dixit Dóminus ad Móysen et Aaron in terra Ægýpti: «Mensis iste vobis princípium ménsium, primus erit in ménsibus anni. 
Loquímini ad univérsum cœtum filiórum Israel et dícite eis:

Décima die mensis huius tollat unusquísque agnum per famílias et domos suas. Sin autem minor est númerus, ut suffícere possit ad vescéndum agnum, assúmet vicínum suum, qui iunctus est dómui suæ, iuxta númerum animárum, quæ suffícere possunt ad esum agni. 
Erit autem vobis agnus absque mácula, másculus, annículus; quem de agnis vel hædis tollétis et servábitis eum usque ad quartam décimam diem mensis huius; immolabítque eum univérsa congregátio filiórum Israel ad vésperam. 

Et sument de sánguine eius ac ponent super utrúmque postem et in superlimináribus domórum, in quibus cómedent illum; et edent carnes nocte illa assas igni et ázymos panes cum lactúcis amáris. 

Sic autem comedétis illum: renes vestros accingétis, calceaménta habébitis in pédibus, tenéntes báculos in mánibus, et comedétis festinánter; est enim Pascha (id est Tránsitus) Dómini!

Et transíbo per terram Ægýpti nocte illa percutiámque omne primogénitum in terra Ægýpti ab hómine usque ad pecus; et in cunctis diis Ægýpti fáciam iudícia, ego Dóminus. Erit autem sanguis vobis in signum in ǽdibus, in quibus éritis; et vidébo sánguinem et transíbo vos, nec erit in vobis plaga dispérdens, quando percússero terram Ægýpti. 

Habébitis autem hanc diem in monuméntum et celebrábitis eam sollémnem Dómino in generatiónibus vestris cultu sempitérno».

\textbf{\Vbar.} Verbum Dómini.
\textbf{\Rbar.} Deo grátias.
\par