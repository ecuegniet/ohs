\begin{small}
\textbf{\Vbar.} Jube, domne, benedícere.

\nadpisleft{Benedictio:} 
Christus perpétuæ det nobis gáudia vitæ.
\textbf{\Rbar.} Amen.
\end{small}

\nadpisleft{Lectio 5 }

Jacóbus frater Dómini, cognoménto Justus, ab ineúnte ætáte vinum et síceram non bibit, carne abstínuit, nunquam tonsus est, nec unguénto nec bálneo usus. 
Huic uni licébat íngredi in Sancta sanctórum. 

Idem líneis véstibus utebátur: cui étiam assidúitas orándi ita callum génibus obdúxerat, ut durítie caméli pellem imitarétur. 

Eum post Christi ascensiónem Apóstoli Jerosolymórum epíscopum creavérunt; \\
ad quem étiam Princeps Apostolórum misit qui nuntiáret se e cárcere ab Angelo edúctum fuísse. 

Cum autem in concílio Jerosólymis controvérsia esset orta de lege et circumcisióne; 

Jacóbus, Petri senténtiam secútus, ad fratres hábuit conciónem, in qua vocatiónem géntium probávit, fratribúsque abséntibus scribéndum esse dixit, ne géntibus jugum Mosáicæ legis impónerent. 

De quo et lóquitur Apóstolus ad Gálatas: \\
Alium autem Apostolórum vidi néminem, nisi Jacóbum fratrem Dómini.

\textbf{\Vbar.} Tu autem, Dómine, miserére nobis.
\textbf{\Rbar.} Deo grátias.

