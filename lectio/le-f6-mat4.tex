\nadpisleft{Lectio 4 (In Psalm 63. ad versum 2.)}

Ex tractátu sancti Augustíni Epíscopi super Psalmos

Protexísti me, Deus, a convéntu malignántium, a multitúdine operántium iniquitátem. Jam ipsum caput nostrum intueamur. Multi Mártyres talia passi sunt, sed nihil sic elucet, quómodo caput Mártyrum: ibi mélius intuemur, quod illi experti sunt. Protectus est a multitúdine malignántium, protegénte se Deo, protegénte carnem suam ipso Fílio, et hómine, quem gerebat: quia fílius hóminis est, et Fílius Dei est. Fílius Dei, propter formam Dei: fílius hóminis, propter formam servi, habens in potestate pónere ánimam suam, et recípere eam. Quid ei potuérunt fácere inimíci? Occidérunt corpus, ánimam non occidérunt. Inténdite. Parum ergo erat, Dóminum hortari Mártyres verbo, nisi firmaret exemplo.
\par