\nadpisleft{Evangelium (Io 20:1-9)}

\textbf{\Vbar.} Dóminus vobíscum.
\textbf{\Rbar.} Et cum spíritu tuo.

\textbf{\Vbar.} Léctio sancti Evangélii secúndum Ioánnem.
\textbf{\Rbar.} Glória tibi, Dómine.

Prima sabbatórum María Magdaléne venit mane, cum adhuc ténebræ essent, ad monuméntum et videt lápidem sublátum a monuménto. 
Currit ergo et venit ad Simónem Petrum et ad álium discípulum, quem amábat Iesus, et dicit eis: «Tulérunt Dóminum de monuménto, et nescímus, ubi posuérunt eum!».

Exiit ergo Petrus et ille álius discípulus, et veniébant ad monuméntum.
Currébant autem duo simul, et ille álius discípulus præcucúrrit cítius Petro et venit primus ad monuméntum; et cum se inclinásset, videt pósita linteámina, non tamen introívit.

Venit ergo et Simon Petrus sequens eum et introívit in monuméntum; et videt linteámina pósita et sudárium, quod fúerat super caput eius, non cum linteamínibus pósitum, sed separátim involútum in unum locum.

Tunc ergo introívit et alter discípulus, qui vénerat primus ad monuméntum, et vidit et crédidit. 
Nondum enim sciébant Scriptúram quia opórtet eum a mórtuis resúrgere.

\textbf{\Vbar.} Verbum Dómini.
\textbf{\Rbar.} Laus tibi, Christe.
\par