\nadpisleft{Lectio 6 (Bar 3:9-15;3:32--4:4)}

Léctio libri Baruch prophétæ

Audi, Israel, mandáta vitæ; áuribus percípite, ut sciátis prudéntiam.
Quid est, Israel? Quid est quod in terra es inimicórum? Inveterásti in terra aliéna, coinquinátus es mórtuis, reputátus es cum eis, qui apud ínferos sunt.
Dereliquísti fontem sapiéntiæ! 
Si in via Dei ambulásses, habitásses in pace in ætérnum.
Disce, ubi sit prudéntia, ubi fortitúdo, ubi sit intelléctus, ut scias simul, ubi sit longitúrnitas diérum et vita, ubi sit lumen oculórum et pax.
Quis invénit locum eius? Et quis intrávit in thesáuros eius?

Sed qui scit ómnia, novit eam, adinvénit eam prudéntia sua; qui compósuit terram in ætérnum tempus, implévit eam iuméntis quadrupédibus; qui mittit lumen et vadit, vocávit illud, et obœdívit ei in tremóre.
Stellæ autem splenduérunt in custódiis suis et lætátæ sunt.
Vocávit eas, et dixérunt: «Adsumus»; luxérunt cum lætítia ei, qui fecit eas.

Hic est Deus noster, non æstimábitur alter advérsus eum.
Invénit omnem viam disciplínæ et dedit eam Iacob púero suo et Israel dilécto suo. Post hæc super terram visa est et inter hómines conversáta est.
Ipsa est liber præceptórum Dei et lex, quæ pérmanet in ætérnum. Omnes, qui tenent eam, ad vitam; qui autem relínquunt eam, moriéntur.

Convértere, Iacob, et apprehénde eam; perámbula ad splendórem in lúmine eius. Noli dare álteri glóriam tuam et dignitátes tuas genti aliénæ.

Beáti sumus, Israel, quia, quæ placent Deo, nobis nota sunt.

\textbf{\Vbar.} Verbum Dómini.
\textbf{\Rbar.} Deo grátias.
\par