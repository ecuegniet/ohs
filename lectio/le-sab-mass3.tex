\nadpisleft{Lectio 3 (Ex 14:15--15:1)}

Léctio libri Exodi

In diébus illis: Dixit Dóminus ad Móysen: «Quid clamas ad me? Lóquere fíliis Israel, ut proficiscántur. Tu autem éleva virgam tuam et exténde manum tuam super mare et dívide illud, ut gradiántur fílii Israel in médio mari per siccum.
Ego autem indurábo cor Ægyptiórum, ut persequántur eos; et glorificábor in pharaóne et in omni exércitu eius, in cúrribus et in equítibus illíus.
Et scient Ægýptii quia ego sum Dóminus, cum glorificátus fúero in pharaóne, in cúrribus atque in equítibus eius».

Tollénsque se ángelus Dei, qui præcedébat castra Israel, ábiit post eos; et cum eo páriter colúmna nubis, prióra dimíttens, post tergum.
Stetit inter castra Ægyptiórum et castra Israel; et erat nubes tenebrósa et illúminans noctem, ita ut ad se ínvicem toto noctis témpore accédere non valérent.

Cumque extendísset Móyses manum super mare, réppulit illud Dóminus, flante vento veheménti et urénte tota nocte, et vertit in siccum; divisáque est aqua.
Et ingréssi sunt fílii Israel per médium maris sicci; erat enim aqua quasi murus a dextra eórum et læva. Persequentésque Ægýptii ingréssi sunt post eos, omnis equitátus pharaónis, currus eius et équites per médium maris.

Iamque advénerat vigília matutína, et ecce respíciens Dóminus super castra Ægyptiórum per colúmnam ignis et nubis perturbávit exércitum eórum; et impedívit rotas cúrruum, ita ut diffícile moveréntur. Dixérunt ergo Ægýptii: «Fugiámus Israélem! Dóminus enim pugnat pro eis contra nos».

Et ait Dóminus ad Móysen: «Exténde manum tuam super mare, ut revertántur aquæ ad Ægýptios super currus et équites eórum». 
Cumque extendísset Móyses manum contra mare, revérsum est primo dilúculo ad priórem locum; fugientibúsque Ægýptiis occurrérunt aquæ, et invólvit eos Dóminus in médiis flúctibus.
Revers\'{\ae}que sunt aquæ et operuérunt currus et équites cuncti exércitus pharaónis, qui sequéntes ingréssi fúerant mare; ne unus quidem supérfuit ex eis.
Fílii autem Israel perrexérunt per médium sicci maris, et aquæ eis erant quasi pro muro a dextris et a sinístris. 

Liberavítque Dóminus in die illo Israel de manu Ægyptiórum. Et vidérunt Ægýptios mórtuos super litus maris et manum magnam, quam exercúerat Dóminus contra eos; timuítque pópulus Dóminum et credidérunt Dómino et Móysi servo eius.

Tunc cécinit Móyses et fílii Israel carmen hoc Dómino, et dixérunt:

%\textbf{\Vbar.} Verbum Dómini.
%\textbf{\Rbar.} Deo grátias.
\par